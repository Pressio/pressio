The functionalities inside the C++ library are organized into several packages\+:


\begin{DoxyItemize}
\item \href{https://github.com/Pressio/pressio/tree/master/packages/mpl/src}{\texttt{ `mpl'}}\+: metaprogramming functionalities;
\item {\ttfamily utils}\+: common functionalities, e.\+g., I/O helpers, static constants, etc;
\item {\ttfamily containers}\+: data strutures wrappers;
\item {\ttfamily ops}\+: linear algebra;
\item {\ttfamily apps}\+: suites of mini-\/apps used for basic testing;
\item {\ttfamily qr}\+: QR factorization functionalities;
\item {\ttfamily svd}\+: singular value decomposition (S\+VD) functionalities;
\item {\ttfamily solvers}\+: linear and non-\/linear solvers (e.\+g. Gauss-\/\+Newton with and without line-\/search, etc.);
\item {\ttfamily ode}\+: explicit and implict time steppers and integrators;
\item {\ttfamily rom}\+: reduced-\/order modeling algorithms.
\end{DoxyItemize}

The top-\/down order used above is informative of the packages\textquotesingle{} dependency structure and mutual visibility. For example, every package depends on {\ttfamily mpl}, but {\ttfamily qr} depends only on {\ttfamily mpl}, {\ttfamily utils}, {\ttfamily containers}. At the bottom of the hierarchy we have the {\ttfamily rom} package which requires all the others. Each package contains corresponding unit-\/ and regular tests. Splitting the framework into separate packages has several benefits.
\begin{DoxyItemize}
\item Maintability\+: {\ttfamily pressio} can be more easily developed and maintained since packages depend on one another through well-\/defined public interfaces, and appropriate namespaces are used to organize classes within each package.
\item Selective usability\+: This modular framework allows users, if needed, to leverage invidual packages. For instance, if a user needs/wants just the QR methods, they simply use that package, and all the dependencies are enabled automatically. 
\end{DoxyItemize}