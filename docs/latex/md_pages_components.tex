The pressio C++ library is divided into several components\+:

\tabulinesep=1mm
\begin{longtabu}spread 0pt [c]{*{4}{|X[-1]}|}
\hline
\PBS\centering \cellcolor{\tableheadbgcolor}\textbf{ Name   }&\PBS\centering \cellcolor{\tableheadbgcolor}\textbf{ Brief Description   }&\PBS\centering \cellcolor{\tableheadbgcolor}\textbf{ Link   }&\PBS\centering \cellcolor{\tableheadbgcolor}\textbf{ Reference header(s)    }\\\cline{1-4}
\endfirsthead
\hline
\endfoot
\hline
\PBS\centering \cellcolor{\tableheadbgcolor}\textbf{ Name   }&\PBS\centering \cellcolor{\tableheadbgcolor}\textbf{ Brief Description   }&\PBS\centering \cellcolor{\tableheadbgcolor}\textbf{ Link   }&\PBS\centering \cellcolor{\tableheadbgcolor}\textbf{ Reference header(s)    }\\\cline{1-4}
\endhead
mpl   &metaprogramming functionalities   &\href{https://github.com/Pressio/pressio/tree/main/include/mpl}{\texttt{ Source}}   &{\ttfamily \#include$<$pressio\+\_\+mpl.\+hpp$>$}    \\\cline{1-4}
utils   &common functionalities~\newline
e.\+g., I/O helpers, static constants, etc   &\href{https://github.com/Pressio/pressio/tree/main/include/utils}{\texttt{ Source}}   &{\ttfamily \#include$<$pressio\+\_\+utils.\+hpp$>$}    \\\cline{1-4}
type\+\_\+traits   &traits/detection classes   &\href{https://github.com/Pressio/pressio/tree/main/include/type_traits}{\texttt{ Source}}   &{\ttfamily \#include$<$pressio\+\_\+type\+\_\+traits.\+hpp$>$}    \\\cline{1-4}
expressions   &expression classes for useful abstractions (span, diagonal, subspan, etc.)   &\href{https://github.com/Pressio/pressio/tree/main/include/expressions}{\texttt{ Source}}   &{\ttfamily \#include$<$pressio\+\_\+expressions.\+hpp$>$}    \\\cline{1-4}
ops   &shared-\/memory and distributed linear algebra kernels specializations   &\href{https://github.com/Pressio/pressio/tree/main/include/ops}{\texttt{ Source}}   &{\ttfamily \#include$<$pressio\+\_\+ops.\+hpp$>$}    \\\cline{1-4}
qr   &QR factorization functionalities   &\href{https://github.com/Pressio/pressio/tree/main/include/qr}{\texttt{ Source}}   &{\ttfamily \#include$<$pressio\+\_\+qr.\+hpp$>$}    \\\cline{1-4}
solvers\+\_\+linear   &linear solvers (wrappers around existing TPLs)   &\href{https://github.com/Pressio/pressio/tree/main/include/solvers_linear}{\texttt{ Source}}   &{\ttfamily \#include$<$pressio\+\_\+solvers\+\_\+linear.\+hpp$>$}    \\\cline{1-4}
solvers\+\_\+nonlinear   &non-\/linear solvers ~\newline
 (e.\+g., Newton-\/\+Raphson, Gauss-\/\+Newton, Levenberg-\/\+Marquardt)   &\href{https://github.com/Pressio/pressio/tree/main/include/solvers_nonlinear}{\texttt{ Source}}   &{\ttfamily \#include$<$pressio\+\_\+solvers\+\_\+nonlinear.\+hpp$>$}    \\\cline{1-4}
ode   &explicit only methods ~\newline
implict only methods ~\newline
 all   &~\newline
~\newline
\href{https://github.com/Pressio/pressio/tree/main/include/ode}{\texttt{ Source}}   &{\ttfamily \#include$<$pressio\+\_\+ode\+\_\+explicit.\+hpp$>$}~\newline
 {\ttfamily \#include$<$pressio\+\_\+ode\+\_\+implicit.\+hpp$>$} ~\newline
 {\ttfamily \#include$<$pressio\+\_\+ode.\+hpp$>$}    \\\cline{1-4}
rom   &Galerkin ROMs ~\newline
 LSPG ROMs ~\newline
 WLS ROMs ~\newline
 all   &~\newline
~\newline
~\newline
\href{https://github.com/Pressio/pressio/tree/main/include/rom}{\texttt{ Source}}   &{\ttfamily \#include$<$pressio\+\_\+rom\+\_\+galerkin.\+hpp$>$} ~\newline
 {\ttfamily \#include$<$pressio\+\_\+rom\+\_\+lspg.\+hpp$>$} ~\newline
 {\ttfamily \#include$<$pressio\+\_\+rom\+\_\+wls.\+hpp$>$} ~\newline
 {\ttfamily \#include$<$pressio\+\_\+rom.\+hpp$>$}   \\\cline{1-4}
\end{longtabu}


The top-\/down order used above is informative of the dependency structure. For example, every package depends on {\ttfamily mpl}. The {\ttfamily ops} package depends only on {\ttfamily mpl}, {\ttfamily utils}, {\ttfamily containers}. At the bottom of the stack we have the {\ttfamily rom} package which requires all the others.

This structure of the framework has several benefits.
\begin{DoxyItemize}
\item Maintability\+: {\ttfamily pressio} can be more easily developed and maintained since its components depend on one another through well-\/defined public interfaces, and appropriate namespaces are used to organize classes.
\item Selective usability\+: this modular framework allows users, if needed, to leverage invidual components. For instance, if a user needs/wants just the QR methods, they can simply use that package, and all the dependencies on the others are enabled automatically.
\end{DoxyItemize}



\begin{DoxyParagraph}{}
When you use functionalities from a specific package, you should just include the corresponding header and the dependencies (based on the explanation above) are included automatically. For example, if you want to do Galerkin with explicit time integration, you just do {\ttfamily \#include $<$pressio\+\_\+rom\+\_\+galerkin.\+hpp$>$} because all the needed packages are automatically included. There is not need to manually include all of them yourself. In the future, we might refine further the granularity of the headers to allow a finer control.
\end{DoxyParagraph}


\begin{DoxyParagraph}{One header to include them all}
If you want to access {\itshape all} functionalities, you can use\+: 
\begin{DoxyCode}{0}
\DoxyCodeLine{\textcolor{preprocessor}{\#include "{}pressio.hpp"{}}}

\end{DoxyCode}
 
\end{DoxyParagraph}
