The adapter class sits between pressio and the external application. This adapter class---if needed---must meet a specific API.

Pressio supports two variants of the API, a so-\/called {\itshape continuous-\/time} and {\itshape discrete-\/time} version. The continuous-\/time API operates such that the user is responsible to compute the continuous-\/time operators, e.\+g., the velocity, and pressio assembles the discrete-\/time operators. It is an API that very expressive of the formulation on which pressio relies on. The discrete-\/time API is designed such that the user is given the necessary operators and operands needed to assemble the time-\/discrete operators directly.

todo\+: why are we forced to keep these two separate? explain

todo\+: discuss pros and cons of the two APIs

The following table illustrates what API is supported for each ROM

\tabulinesep=1mm
\begin{longtabu}spread 0pt [c]{*{3}{|X[-1]}|}
\hline
\PBS\centering \cellcolor{\tableheadbgcolor}\textbf{ ROM Method   }&\PBS\centering \cellcolor{\tableheadbgcolor}\textbf{ Continuous-\/time API support   }&\PBS\centering \cellcolor{\tableheadbgcolor}\textbf{ Discrete-\/time API support    }\\\cline{1-3}
\endfirsthead
\hline
\endfoot
\hline
\PBS\centering \cellcolor{\tableheadbgcolor}\textbf{ ROM Method   }&\PBS\centering \cellcolor{\tableheadbgcolor}\textbf{ Continuous-\/time API support   }&\PBS\centering \cellcolor{\tableheadbgcolor}\textbf{ Discrete-\/time API support    }\\\cline{1-3}
\endhead
Galerkin (explicit time stepping)   &yes   &no    \\\cline{1-3}
Galerkin (implicit time stepping)   &yes   &yes    \\\cline{1-3}
LSPG   &yes   &yes    \\\cline{1-3}
WLS   &yes   &yes   \\\cline{1-3}
\end{longtabu}
