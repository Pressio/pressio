\#\+As discussed in \href{./md_pages_getstarted_pressio_app.html}{\texttt{ this}} section of the get started page, The adapter class sits between pressio and the external application. This adapter class---if needed---must meet a specific A\+PI.

Pressio supports two variants of the A\+PI, a so-\/called {\itshape continuous-\/time} and {\itshape discrete-\/time} version. The continuous-\/time A\+PI operates such that the user is responsible to compute the continuous-\/time operators, e.\+g., the velocity, and pressio assembles the discrete-\/time operators. It is an A\+PI that very expressive of the formulation on which pressio relies on. The discrete-\/time A\+PI is designed such that the user is given the necessary operators and operands needed to assemble the time-\/discrete operators directly.

todo\+: why are we forced to keep these two separate? explain

todo\+: discuss pros and cons of the two A\+P\+Is

The following table illustrates what A\+PI is supported for each R\+OM

\tabulinesep=1mm
\begin{longtabu}spread 0pt [c]{*{3}{|X[-1]}|}
\hline
\PBS\centering \cellcolor{\tableheadbgcolor}\textbf{ R\+OM Method  }&\PBS\centering \cellcolor{\tableheadbgcolor}\textbf{ Continuous-\/time A\+PI support  }&\PBS\centering \cellcolor{\tableheadbgcolor}\textbf{ Discrete-\/time A\+PI support   }\\\cline{1-3}
\endfirsthead
\hline
\endfoot
\hline
\PBS\centering \cellcolor{\tableheadbgcolor}\textbf{ R\+OM Method  }&\PBS\centering \cellcolor{\tableheadbgcolor}\textbf{ Continuous-\/time A\+PI support  }&\PBS\centering \cellcolor{\tableheadbgcolor}\textbf{ Discrete-\/time A\+PI support   }\\\cline{1-3}
\endhead
Galerkin (explicit time stepping)  &yes  &no   \\\cline{1-3}
Galerkin (implicit time stepping)  &yes  &yes   \\\cline{1-3}
L\+S\+PG  &yes  &yes   \\\cline{1-3}
W\+LS  &yes  &yes   \\\cline{1-3}
\end{longtabu}
